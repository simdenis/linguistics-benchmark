\section{Isomorphic Variants and Leakage Measurement}
\label{sec:isomorphs}

\paragraph{Definition.}
Given an item $x$ with gold answer $y$, an \emph{isomorphic variant} is a transformation $T$ that produces $x' = T(x)$ such that there exists a deterministic induced mapping $T_y$ with $y' = T_y(y)$ and $(x', y')$ preserves the underlying reasoning structure.
In short: the puzzle is ``the same,'' but surface strings change.

\paragraph{Transformations.}
We use compositions of:
\begin{itemize}
  \item \textbf{Label permutation:} permute the letter labels of multiple-choice or matching lists.
  \item \textbf{Example reordering:} reorder the presentation of examples without changing content.
  \item \textbf{Lexeme renaming:} consistently rename the unknown-language tokens and/or English gloss words via a bijection.
  \item \textbf{Format-preserving noise:} whitespace and punctuation perturbations that do not alter the intended reading.
\end{itemize}
We avoid transformations that change linguistic difficulty (e.g., adding/removing distractors).

\paragraph{Memorization gap.}
We define
\[
\gap \;=\; \mathrm{Acc}(\orig) - \mathrm{Acc}(\iso),
\]
where $\mathrm{Acc}(\orig)$ is accuracy on original items and $\mathrm{Acc}(\iso)$ is accuracy on isomorphic variants.
A large positive \gap suggests sensitivity to surface form consistent with memorization or brittle pattern matching.

\paragraph{Variant generation protocol.}
For each test item, we generate $k$ variants (default $k=3$) with fixed random seeds for reproducibility.
\TODO{Add details: which spans are variantable and how you sample bijections.}
